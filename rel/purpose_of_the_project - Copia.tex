\chapter{Scopo del Progetto} 
 
\section{Abstract}

E' stata sviluppata un'applicazione Desktop denominata Pets and Owners (da qui in poi PAO). 
L'applicazione si prefigge di modellare e gestire, molto semplicemente, la parte amministrativa di un piccolo database di una clinica veterinaria fornendo un'interfaccia utente semplice ed intuitiva sviluppata in \textit{Qt}.

\section{Note per la compilazione}
Per compilare il progetto, viene fornito un file .pro. E' necessario utilizzare la versione 5.3.2 o successiva delle librerie \textit{Qt} seguendo le istruzioni riportate all'url  \footnotetext[1]{ \url{http://www.studenti.math.unipd.it/index.php?id=corsi\#c688} }
\begin{itemize}
	\item lanciare lo script per la compilazione \textit{qt-532.sh} (non incluso nella consegna)
	\item Posizionarsi nella cartella del progetto contente i sorgenti
	\item lanciare \textit{qmake} 
	\item \textit{lanciare make ./cartellaDiBuild} , cartellaDiBuild conterrà l'eseguibile. Vengono inoltre forniti 3 file \textit{.json} che compongono un piccolo DB dimostrativo da inserire nella stessa cartella contenente il file eseguibile. 
\end{itemize}

L'applicazione è stata sviluppata usando:
\begin{itemize}
	\item QtCreator 3.2.1
	\item Qt 5.3.2
	\item GCC 4.8.6 e GCC 5.3.0
	\item Ubuntu 16.04 Xenial Xerus 64 Bit - Windows 7 Professional 64 Bit aggiungendo le direttive per QMake \textit{ INCLUDEPATH += .\ e  DEPENDPATH += .\ } Su Windows la build viene eseguita correttamente da QtCreator ma usando MinGW da linea di comando si potrebbe incappare in un'errore \footnotetext[2]{ \url{https://stackoverflow.com/questions/12573816/what-is-an-undefined-reference-unresolved-external-symbol-error-and-how-do-i-fix} }

\end{itemize}

\section{Vincoli obbligatori}
\begin{itemize}
\item \textbf{Definizione ed utilizzo di una gerarchia G di tipi di altezza $\geq 1$ e larghezza $\geq 1$} A tal proposito è stata sviluppata una gerarchia di Visite, descritta in seguito.
\item \textbf{ Definizione di un opportuno contenitore C, con relativi iteratori, che permetta inserimenti, rimozioni, modifiche.} E' stato sviluppata una classe \textit{PAOContainer} per la gestione delle visite che verrà illustrata nella sezione \textit{Orientamento agli oggetti}
\item \textit{Utilizzo del contenitore C per memorizzare oggetti polimorfi della gerarchia G.} Il \textbf{PAOContainer} memorizza e gestisce puntatori polimorfi di \textit{AbstractVisit}, l'uso di puntatori è necessario per utilizzare polimorfismo e i metodi virtuali descritti in seguito.
\item \textbf{Il front-end dell’applicazione è una GUI sviluppata usando il framework Qt.}
\end{itemize}
\