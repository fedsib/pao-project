\chapter{Considerazioni Finali} 

\section{Note per l'installazione}

Per permettere un'agevole navigazione nel progetto sono stati aggiunti due account forniti di diverso livello di permessi. \\
\begin{itemize}
\item E' possibile l'accesso all'account  di un utente socio generico con permessi di default usando le credenziali:
\begin{itemize}
\item \textbf{\textit{Username:}} user
\item \textbf{\textit{Password:}} user
\end{itemize} 

\item E' possibile l'accesso all'account di un utente istruttore generico con permessi \\ amministrativi usando le credenziali:
\begin{itemize}
\item \textbf{\textit{Username:}} admin
\item \textbf{\textit{Password:}} admin
\end{itemize} 
\end{itemize}

Sempre per comodita' di valutazione, e' possibile l'accesso come un qualsiasi utente visto che le password coincodono coincidono con lo username dell'account.

\section{Licenza}
Tutto il codice sorgente in PHP, CSS, SQL compreso quello di questa relazione \LaTeX  è da intendersi come rilasciato sotto \textit{GPLv3}\\
\url{http://choosealicense.com/licenses/gpl-3.0/} 
 . \\ Il file risultante dalla compilazione del sorgente \LaTeX è da considerarsi come rilasciato sotto licenza \textit{CC BY-SA 4.0}\\ \url{http://creativecommons.org/licenses/by-sa/4.0/} 

\section{Strumenti utilizzati}
Per la realizzazione degli schemi è stato utilizzato il software \textbf{Gnome Dia}, come strumenti di collaborazione son stati utilizzati un repository su \textit{github} e una cartella su cloud presso il servizio di cloud storage \textit{mega.nz}.