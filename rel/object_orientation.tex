\chapter{Orientamento agli Oggetti} 
 
\section{Descrizione delle Gerarchie}

L'intera applicazione si compone di 20 classi (12 per le gerarchie principali, 3 di tipi utilità e 6 per la GUI).
La gerarchia principale, nonché la più importante è la gerarchia delle visite. Essa si compone di quattro classi:
\begin{itemize}
	\item \textit{\textbf{AbstractVisit}}: Una classe virtuale pura (non istanziabile direttamente), contiene 3 metodi virtuali puri, uno per scrivere su file i campi contenuti nelle classi derivate, uno per calcolare il prezzo finale della visita e uno di utilità che ritorna una stringa indicante il tipo di Visita.
	\item \textit{\textbf{BasicVisit}}: Deriva pubblicamente da \textit{AbstractVisit} e calcola il costo finale, \textit{override} dell'apposito metodo virtuale puro, restituendo il prezzo base.
	\item \textit{\textbf{VaccineVisit}} Deriva pubblicamente da \textit{AbstractVisit} e calcola il costo finale tramite override,  come il doppio del prezzo base di una visita di base.
	\item \textit{\textbf{SpecializedVisit}} Deriva pubblicamente da \textit{AbstractVisit} e calcola il costo finale con un 30 \% aggiuntivo rispetto al prezzo di un Vaccino.
\end{itemize} 

\subsection{Utilità e ruolo dei tipi}
 Oltre alle quattro classi descritte sopra, ve ne sono altre:
 \begin{itemize}
 	\item \textit{\textbf{AbstractOwner}}: Modella un generico proprietario di animali (può essere un veterinario), è una classe virtuale pura, madre della seconda gerarchia. Contiene due metodi virtuali puri, usati per restituire una stringa e per salvare i dati su file; contiene inoltre un puntatore ad una classe \textit{OwnerAccount} e una stringa per memorizzare il Codice Fiscale.
 	\item \textit{\textbf{Owner}}: Deriva pubblicamente da \textit{AbstractOwner} e rappresenta un proprietario privato di animali. Contiene un puntatore ad una classe \textit{OwnerData}.
 	\item \textit{\textbf{OwnerAccount}}: Utilizzata per funzionalità future. Permette di distinguere se un utente è o no un veterinario. Inoltre fornisce delle funzionalità di Encryption della password, memorizzata tramite un hash.
 	\item \textit{\textbf{OwnerData}}: Contiene le informazioni di un proprietario, quali nome, cognome e informazioni di contatto, queste ultime tramite un campo dati di tipo \textit{ContactInfo}
 	\item \textit{\textbf{ContactInfo}}: Utilizzata in \textit{OwnerData}, descrive indirizzo fisico, mail e numero di telefono di un proprietario.
 	\item \textit{\textbf{Animal}}: Descrive un animale generico. E' una classe virtuale pura, madre della terza gerarchia utilizzata dall'applicazione.
 	Contiene due metodi virtuali puri per restituire una stringa e per salvare i campi dati su file. Contiene inoltre un campo ad una classe \textit{AnimalData}
 	\item \textit{\textbf{Pet}}: Estende pubblicamente \textit{Animal} e rappresenta un animale domestico, caratterizzato dal tipo (Cane, Gatto, Volatile o Altro), dal colore del pelo e dalla razza dell'animale.
 	\item \textit{\textbf{AnimalData}}:Contiene le informazioni essenziali di un animale, quali nome, data di nascita, sesso e peso ed eventualmente data di applicazione del microchip.
 \end{itemize}
 
 Vi sono poi due classi d'utilità: Il contenitore delle visite e il database vero e proprio. Le ulteriori sei classi che modellano la parte grafica verranno trattate nella sezione \textit{Interfaccia Grafica}.
 
 \begin{itemize}
 	\item \textit{\textbf{PaoDB}}: Gestisce gli utenti tramite una \textit{QMap<QString, AbstractOwner*>}, gli animali tramite una \textit{QMap<QString, Animal*>} e le visite tramite il contenitore preposto, \textit{PaoContainer}. \\ La classe fornisce metodi per inserire, modificare, rimuovere proprietari, animali e visite, per leggere i dati delle tre entità da file e caricarli in memoria e per salvarli su disco rigido in formato \textit{JSON}
 \end{itemize}
 
 
 \subsection{Descrizione del contenitore}
 La classe PaoContainer altro non è che una lista che permette inserimenti in testa ed in coda. \\Essa si avvale di due classi interne private: \textit{node}, la lista vera e propria di nodi e \textit{SmartPointer} per la gestione della memoria condivisa, così come visto durante il corso. 
 \\ Il contenitore mette a disposizione per iterare sui suoi elementi due tipi di iteratori: Un \textit{iterator} per accedere normalmente agli elementi del contenitore e un \textit{const\_iterator} utilizzato per la sola lettura. La classe \textit{PaoContainer} è amica degli iteratori e viceversa, questo permette agli iteratori di accedere alla parte privata di \textit{PaoContainer} e a \textit{PaoContainer} di accedere alla parte privata degli iteratori. \\ 
 
 \textit{PaoContainer} fornisce funzioni per determinare la sua dimensione (\textit{size()}), per verificare se è vuoto (\textit{isEmpty()}) per l'inserimento in testa (\textit{push\_front}) e in coda (\textit{push\_back}), per ottenere una visita da modificare, per ritornare l'iteratore iniziale e l'iteratore \textit{Past the End} e per rimuovere visite.
 I suoi due campi dati sono uno \textit{SmartPointer} che punta alla testa del Contenitore e un intero che ne rappresenta la size, cioè il numero di elementi contenuti. Container fornisce inoltre l'\textit{overloading} dell'operatore di accesso a membro \textit{operator[]}, altri operatori sono stati ridefiniti nelle classi interne.
 
\section{Incapsulamento}

Si è scelto di garantire un incapsulamento fornendo alle varie classi, alcune funzioni cercando di usare, quanto più, metodi costanti e  gli specificatori di accesso \textit{private} e \textit{protected} rendendo \textit{public} solo dei metodi \textit{getter} e \textit{setter} di interfaccia.
\\Questo, da una parte permette di non accedere direttamente ai campi dati ma dall'altra espone in parte la struttura delle classi.
Alla fine, dopo averci riflettuto su a lungo, per quanto la soluzione non sia delle migliori e visti gli scopi del progetto è risultata essere la scelta più conveniente. \footnotetext[3]{ \url{http://goo.gl/CPF3up} } \footnotetext[4]{ \url{https://dzone.com/articles/getter-setter-use-or-not-use-0} }

Un incapsulamento soddisfacente è stato ottenuto per il salvataggio dei dati delle singole classi su file, dove ogni classe è responsabile di salvare i propri campi dati.

\section{Modularità}

Ogni classe è stata suddivisa in un file di header .h contenente le dichiarazioni e in un file .cpp contenente le definizioni implementative. Si è cercato inoltre di suddividere i compiti in varie funzioni secondo il principio della \textit{Single-Responsibility}, ovvero un compito specifico per ciascuna funzione.

\section{Estensibilità $\backslash$ Evolvibilità}

Si è scelto di far partire le tre gerarchie con delle classi astratte, in modo da permettere una eventuale estensione futura del software. 
Ad esempio, si potranno creare dei diversi tipi di proprietario derivando da \textit{AbstractOwner}, per rappresentare uno zoo o un parco nazionale. 
Si potrà derivare \textit{Animal} per nuove categorie di animali (esotici, selvaggi, da giardino) e soprattutto si potranno creare nuovi tipi di visite a partire da \textit{AbstractVisit}, includendo, ad esempio, un accertamento radiologico con un prezzo appositamente calcolato.
\\ 

\subsection{Esempi d'uso di codice polimorfo}
\lstinputlisting[language=C++, firstline=6, lastline=9]{cpp/specializedvisit.cpp} 
Un esempio di \textit{override} di un metodo virtuale puro, per il calcolo del costo finale di una visita specialistica.\\

\lstinputlisting[language=C++, firstline=120, lastline=135]{cpp/vetcontrol.cpp} 

Un esempio di RTTI per estrarre i dati dei proprietari da aggiungere alla tabella.\\

\lstinputlisting[language=C++, firstline=14, lastline=18]{cpp/specializedvisit.cpp} 

Sempre sulla visita specialistica, un esempio di \textit{override} per salvare i dati su file.\\

\textit{\textbf{Nota:}} Si voleva utilizzare il marcatore esplicito \textit{override} previsto da \textit{C++11}, in modo da rendere più chiaro quali metodi sono degli override all'interno delle classi, tuttavia ciò non è stato possibile in quanto la versione di GCC presente in laboratorio non supportava ancora tale funzione.

\section{Efficienza $\backslash$ Robustezza}
Si è cercato di rendere robusta soprattutto l'interfaccia grafica tramite dei controlli elementari sui campi dati inseriti, ad esempio le chiavi non possono essere una \textit{QString} vuota.